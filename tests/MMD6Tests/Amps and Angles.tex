\input{mmd6-article-leader}
\def\mytitle{Amps and Angles}
\input{mmd6-article-begin}

AT\&T has an ampersand in their name.

AT\&T is another way to write it.

This \& that.

4 < 5.

6 > 5.

5

Here is a \href{http://example.com/?foo=1&bar=2}{link}\footnote{\href{http://example.com/?foo=1&bar=2}{http:\slash \slash example.com\slash ?foo=1\&bar=2}} with an ampersand in the URL.

Here is a link with an amersand in the link text: \href{http://att.com/}{AT\&T}\footnote{\href{http://att.com/}{http:\slash \slash att.com\slash }}.

Here is an inline \href{/script%20here?foo=1&bar=2}{link}\footnote{\href{/script%20here?foo=1&bar=2}{\slash script\%20here?foo=1\&bar=2}}.

Here is an inline \href{/script%20here?foo=1&bar=2}{link}\footnote{\href{/script%20here?foo=1&bar=2}{\slash script\%20here?foo=1\&bar=2}}.

\begin{verbatim}
& and &amp; and < and > in code block.
\end{verbatim}

10

\&copy; \texttt{\&copy;}

\&\#169; \texttt{\&\#169;}

\&\#xA9; \texttt{\&\#xA9;}

15

\texttt{\& and \& and < and > in code span.}

\input{mmd6-article-footer}
\end{document}
