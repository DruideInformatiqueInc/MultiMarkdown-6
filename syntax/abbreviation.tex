\input{mmd6-tufte-handout-leader}
\def\mytitle{Abbreviations}
\def\myauthor{Fletcher T. Penney}
\def\revised{2018-06-27}
\newacronym{MMD}{MMD}{MultiMarkdown}

\newacronym{MD}{MD}{Markdown}

\input{mmd6-tufte-handout-begin}

\section{Abbreviations (or Acronyms)}
\label{abbreviationsoracronyms}

\textbf{NOTE}: The syntax for abbreviations changed in \gls{MMD} v6.

Abbreviations can be specified using inline or reference syntax. The inline variant requires that the abbreviation be wrapped in parentheses and immediately follows the \texttt{>}.

\begin{verbatim}
[>MMD] is an abbreviation.  So is [>(MD) Markdown].

[>MMD]: MultiMarkdown
\end{verbatim}

There is also a ``shortcut'' method for abbreviations that is similar to the approach used in prior versions of \gls{MMD}. You specify the definition for the abbreviation in the usual manner, but \gls{MMD} will automatically identify each instance where the abbreviation is used and substitute it automatically. In this case, the abbreviation is limited to a more basic character set which includes letters, numbers, periods, and hyphens, but not much else. For more complex abbreviations, you must explicitly mark uses of the abbreviation.

There are a few limitations:

\begin{itemize}
\item The full name of the abbreviation is plain text only -- no MultiMarkdown markup will be processed.

\item When exporting to LaTeX, the \texttt{acronym} package is used; this means that the first usage will result in \texttt{full text (short)}, and subsequent uses will result in \texttt{short}.

\end{itemize}

\input{mmd6-tufte-handout-footer}
\end{document}
