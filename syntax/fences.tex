\input{mmd6-tufte-handout-leader}
\def\mytitle{Fenced Code Blocks}
\def\myauthor{Fletcher T. Penney}
\def\revised{2018-06-30}
\newacronym{MMD}{MMD}{MultiMarkdown}

\newacronym{MD}{MD}{Markdown}

\input{mmd6-tufte-handout-begin}

\tableofcontents

\section{Fenced Code Blocks }
\label{fencedcodeblocks}

In addition to the regular indented code block that Markdown uses, you can use ``fenced'' code blocks in MultiMarkdown. These code blocks do not have to be indented, and can also be configured to be compatible with a third party syntax highlighter. These code blocks should begin with 3 to 5 backticks, an optional language specifier (if using a syntax highlighter), and should end with the same number of backticks you started with:

\begin{verbatim}
```perl
# Demonstrate Syntax Highlighting if you link to highlight.js #
# http://softwaremaniacs.org/soft/highlight/en/
print "Hello, world!\n";
$a = 0;
while ($a < 10) {
print "$a...\n";
$a++;
}
```
\end{verbatim}

\begin{lstlisting}[language=perl]
# Demonstrate Syntax Highlighting if you link to highlight.js #
# http://softwaremaniacs.org/soft/highlight/en/
print "Hello, world!\n";
$a = 0;
while ($a < 10) {
print "$a...\n";
$a++;
}
\end{lstlisting}

I don't recommend any specific syntax highlighter, but have used the following metadata to set things up. It may or may not work for you:

\begin{verbatim}
html header:	<link rel="stylesheet" href="http://yandex.st/highlightjs/7.3/styles/default.min.css">
	<script src="http://yandex.st/highlightjs/7.3/highlight.min.js"></script>
	<script>hljs.initHighlightingOnLoad();</script>
\end{verbatim}

Fenced code blocks are particularly useful when including another file (\href{https://fletcher.github.io/MultiMarkdown-6/syntax/transclusion.html}{File Transclusion}\footnote{\href{https://fletcher.github.io/MultiMarkdown-6/syntax/transclusion.html}{https:\slash \slash fletcher.github.io\slash MultiMarkdown-6\slash syntax\slash transclusion.html}}), and you want to show the \emph{source} of the file, not what the file looks like when processed by MultiMarkdown.

\textbf{NOTE}: In MultiMarkdown v6, if there is no closing fence, then the code block continues until the end of the document.

\input{mmd6-tufte-handout-footer}
\end{document}
