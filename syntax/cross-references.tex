\input{mmd6-tufte-handout-leader}
\def\mytitle{Cross-References}
\def\myauthor{Fletcher T. Penney}
\def\revised{2018-06-30}
\newacronym{MMD}{MMD}{MultiMarkdown}

\newacronym{MD}{MD}{Markdown}

\input{mmd6-tufte-handout-begin}

\tableofcontents

\section{Cross-References }
\label{cross-references}

An oft-requested feature was the ability to have Markdown automatically handle
within-document links as easily as it handled external links. To this aim, I
added the ability to interpret \texttt{[Some Text][]} as a cross-link, if a header
named ``Some Text'' exists.

As an example, \texttt{[Metadata][]} will take you to the
{[section describing metadata]}{[Metadata]}.

Alternatively, you can include an optional label of your choosing to help
disambiguate cases where multiple headers have the same title:

\begin{verbatim}
### Overview [MultiMarkdownOverview] ##
\end{verbatim}

This allows you to use \texttt{[MultiMarkdownOverview]} to refer to this section
specifically, and not another section named \texttt{Overview}. This works with atx-
or settext-style headers.

If you have already defined an anchor using the same id that is used by a
header, then the defined anchor takes precedence.

In addition to headers within the document, you can provide labels for images
and tables which can then be used for cross-references as well.

\input{mmd6-tufte-handout-footer}
\end{document}
