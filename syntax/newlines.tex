\input{mmd6-tufte-handout-leader}
\def\mytitle{Escaped Newlines}
\def\myauthor{Fletcher T. Penney}
\def\revised{2018-06-30}
\newacronym{MMD}{MMD}{MultiMarkdown}

\newacronym{MD}{MD}{Markdown}

\input{mmd6-tufte-handout-begin}

\tableofcontents

\section{Escaped newlines}
\label{escapednewlines}

Thanks to a contribution from \href{https://github.com/njmsdk}{Nicolas}\footnote{\href{https://github.com/njmsdk}{https:\slash \slash github.com\slash njmsdk}}, MultiMarkdown has an additional syntax to indicate a line break. The usual approach for Markdown is ``space-space-newline'' --- two spaces at the end of the line. For some users, this causes problems:

\begin{itemize}
\item the trailing spaces are typically invisible when glancing at the source, making it easy to overlook them

\item some users' text editors modify trailing space (IMHO, the proper fix for this is a new text editor{\ldots})

\end{itemize}

Nicolas submitted a patch that enables a new option that interprets ``\textbackslash{}'' before a newline as a marker that a line break should be used:

\begin{verbatim}
This is a line.\
This is a new line.
\end{verbatim}

\input{mmd6-tufte-handout-footer}
\end{document}
