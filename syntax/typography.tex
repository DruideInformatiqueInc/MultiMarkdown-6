\input{mmd6-tufte-handout-leader}
\def\mytitle{Smart Typography}
\def\myauthor{Fletcher T. Penney}
\def\revised{2018-06-27}
\newacronym{MMD}{MMD}{MultiMarkdown}

\newacronym{MD}{MD}{Markdown}

\input{mmd6-tufte-handout-begin}

\tableofcontents

\section{Smart Typography}
\label{smarttypography}

MultiMarkdown incorporates John Gruber's \href{http://daringfireball.net/projects/smartypants/}{SmartyPants}\footnote{\href{http://daringfireball.net/projects/smartypants/}{http:\slash \slash daringfireball.net\slash projects\slash smartypants\slash }} tool in addition to the core Markdown functionality. This program converts ``plain'' punctuation into ``smarter'' typographic punctuation.

Just like the original, MultiMarkdown converts:

\begin{itemize}
\item Straight quotes (\texttt{"} and \texttt{'}) into ``curly'' quotes

\item Backticks-style quotes (\texttt{``this''}) into ``curly'' quotes

\item Dashes (\texttt{-{}-} and \texttt{-{}-{}-}) into en- and em- dashes

\item Three dots (\texttt{...}) become an ellipsis

\end{itemize}

MultiMarkdown also includes support for quotes styles other than English (the default). Use the \texttt{quotes language} metadata to choose:

\begin{itemize}
\item dutch (\texttt{nl})

\item german(\texttt{de})

\item germanguillemets

\item french(\texttt{fr})

\item spanish(\texttt{en})

\item swedish(\texttt{sv})

\end{itemize}

This feature is enabled by default, but is disabled in \texttt{compatibility} mode, since it is not part of the original Markdown. You can also use the \texttt{-{}-nosmart} command line option to disable this feature.

\input{mmd6-tufte-handout-footer}
\end{document}
