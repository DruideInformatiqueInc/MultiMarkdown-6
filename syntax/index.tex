\input{mmd6-tufte-handout-leader}
\def\mytitle{MultiMarkdown Syntax}
\def\myauthor{Fletcher T. Penney}
\def\revised{2017-06-07}
\newacronym{MMD}{MMD}{MultiMarkdown}

\newacronym{MD}{MD}{Markdown}

\input{mmd6-tufte-handout-begin}

\tableofcontents

\chapter{Syntax }
\label{syntax}

\section{Abbreviations (or Acronyms)}
\label{abbreviationsoracronyms}

\textbf{NOTE}: The syntax for abbreviations changed in \gls{MMD} v6.

Abbreviations can be specified using inline or reference syntax. The inline variant requires that the abbreviation be wrapped in parentheses and immediately follows the \texttt{>}.

\begin{verbatim}
[>MMD] is an abbreviation.  So is [>(MD) Markdown].

[>MMD]: MultiMarkdown
\end{verbatim}

There is also a ``shortcut'' method for abbreviations that is similar to the approach used in prior versions of \gls{MMD}. You specify the definition for the abbreviation in the usual manner, but \gls{MMD} will automatically identify each instance where the abbreviation is used and substitute it automatically. In this case, the abbreviation is limited to a more basic character set which includes letters, numbers, periods, and hyphens, but not much else. For more complex abbreviations, you must explicitly mark uses of the abbreviation.

There are a few limitations:

\begin{itemize}
\item The full name of the abbreviation is plain text only -- no MultiMarkdown markup will be processed.

\item When exporting to LaTeX, the \texttt{acronym} package is used; this means that the first usage will result in \texttt{full text (short)}, and subsequent uses will result in \texttt{short}.

\end{itemize}

\section{Citations }
\label{citations}

I have included support for \emph{basic} bibliography features in this version of MultiMarkdown. Please give me feedback on ways to improve this but keep the following in mind:

\begin{enumerate}
\item Bibliography support in MultiMarkdown is rudimentary. The goal is to offer a basic standalone feature, that can be changed using the tool of your choice to a more robust format (e.g. BibTeX, CiteProc). My XSLT files demonstrate how to make this format compatible with BibTeX, but I am not planning on personally providing compatibility with other tools. Feel free to post your ideas and tools to the wiki.

\item Those needing more detailed function sets for their bibliographies may need customized tools to provide those services. This is a basic tool that should work for most people. Reference librarians will probably not be satisfied however.

\end{enumerate}

To use citations in MultiMarkdown, you use a syntax much like that for anchors:

\begin{verbatim}
This is a statement that should be attributed to
its source[p. 23][#Doe:2006].

And following is the description of the reference to be
used in the bibliography.

[#Doe:2006]: John Doe. *Some Big Fancy Book*.  Vanity Press, 2006.
\end{verbatim}

You are not required to use a locator (e.g. p. 23), and there are no special rules on what can be used as a locator if you choose to use one. If you prefer to omit the locator, just use an empty set of square brackets before the citation:

\begin{verbatim}
This is a statement that should be attributed to its 
source[][#Doe:2006].
\end{verbatim}

There are no rules on the citation key format that you use (e.g. \texttt{Doe:2006}), but it must be preceded by a \texttt{\#}, just like footnotes use \texttt{\^{}}.

As for the reference description, you can use Markup code within this section, and I recommend leaving a blank line afterwards to prevent concatenation of several references. Note that there is no way to reformat these references in different bibliography styles; for this you need a program designed for that purpose (e.g. BibTeX).

If you want to include a source in your bibliography that was not cited, you may use the following:

\begin{verbatim}
[Not cited][#citekey]
\end{verbatim}

The \texttt{Not cited} bit is case insensitive.

If you are creating a LaTeX document, the citations will be included, and natbib will be used by default. If you are not using BibTeX and are getting errors about your citations not being compatible with `Author-Year', you can add the following to your documents metadata:

\begin{verbatim}
latex input:	mmd-natbib-plain
\end{verbatim}

This changes the citation style in natbib to avoid these errors, and is useful when you include your citations in the MultiMarkdown document itself.

\textbf{NOTE}: As of version 6, HTML wraps citation references in parentheses instead of brackets, e.g. \texttt{(1)} instead of \texttt{[1]}. Also, citations are now displayed in a separate section from footnotes when outputting to HTML.

\subsection{Inline Citations }
\label{inlinecitations}

Citations can also be used in an inline syntax, just like inline footnotes:

\begin{verbatim}
As per Doe.[#John Doe. *A Totally Fake Book 1*.  Vanity Press, 2006.]
\end{verbatim}

\subsection{BibTeX }
\label{bibtex}

If you are creating a LaTeX document, and need a bibliography, then you should definitely look into \href{http://www.bibtex.org/}{BibTeX}\footnote{\href{http://www.bibtex.org/}{http:\slash \slash www.bibtex.org\slash }} and \href{http://merkel.zoneo.net/Latex/natbib.php}{natbib}\footnote{\href{http://merkel.zoneo.net/Latex/natbib.php}{http:\slash \slash merkel.zoneo.net\slash Latex\slash natbib.php}}. It is beyond the scope of this document to describe how these two packages work, but it is possible to combine them with MultiMarkdown.

To use BibTeX in a MultiMarkdown document, you \emph{must} use the \texttt{BibTeX} metadata (\autoref{bibtex}) to specify where your citations are stored. You may \emph{optionally} use the \texttt{biblio style} metadata key.

Since \texttt{natbib} is enabled by default, you have a choice between using the \texttt{\textbackslash{}citep} and \texttt{\textbackslash{}citet} commands. The following shows how this relates to the MultiMarkdown syntax used.

\begin{verbatim}
[#citekey]    => ~\citep{citekey}
[#citekey][]  => ~\citep{citekey}

[foo][#citekey] => ~\citep[foo]{citekey}

[foo\]\[bar][#citekey] => ~\citep[foo][bar]{citekey}


[#citekey;]    => \citet{citekey}
[#citekey;][]  => \citet{citekey}

[foo][#citekey;] => \citet[foo]{citekey}

[foo\]\[bar][#citekey;] => \citet[foo][bar]{citekey}
\end{verbatim}

\section{CriticMarkup }
\label{criticmarkup}

\subsection{What Is CriticMarkup? }
\label{whatiscriticmarkup}

\begin{quote}
CriticMarkup is a way for authors and editors to track changes to documents in plain text. As with Markdown, small groups of distinctive characters allow you to highlight insertions, deletions, substitutions and comments, all without the overhead of heavy, proprietary office suites. \href{http://criticmarkup.com/}{http:\slash \slash criticmarkup.com\slash }
\end{quote}

CriticMarkup is integrated with MultiMarkdown itself, as well as \href{http://multimarkdown.com/}{MultiMarkdown Composer}\footnote{\href{http://multimarkdown.com/}{http:\slash \slash multimarkdown.com\slash }}. I encourage you to check out the \href{http://criticmarkup.com/}{CriticMarkup}\footnote{\href{http://criticmarkup.com/}{http:\slash \slash criticmarkup.com\slash }} web site to learn more as it can be a very useful tool. There is also a great video showing CriticMarkup in use while editing a document in MultiMarkdown Composer.

\subsection{The CriticMarkup Syntax }
\label{thecriticmarkupsyntax}

The CriticMarkup syntax is fairly straightforward. The key thing to remember is that CriticMarkup is processed \emph{before} any other MultiMarkdown is handled. It's almost like a separate layer on top of the MultiMarkdown syntax.

When editing in MultiMarkdown Composer, you can have CriticMarkup syntax flagged in the both the editor pane and the preview window. This will allow you to see changes in the HTML preview.

When using CriticMarkup with MultiMarkdown itself, you have four choices:

\begin{itemize}
\item Leave the CriticMarkup syntax in place (\texttt{multimarkdown foo.txt}). MultiMarkdown will attempt to show the changes as highlights in the exported document, where possible.

\item Accept all changes, giving you the ``new'' document (\texttt{multimarkdown -a foo.txt} or \texttt{multimarkdown -{}-accept foo.txt})

\item Reject all changes, giving you the ``original'' document (\texttt{multimarkdown -r foo.txt} or \texttt{multimarkdown -{}-reject foo.txt})

\end{itemize}

CriticMarkup comments and highlighting are ignored when processing.

Deletions from the original text:

\begin{verbatim}
This is {--is --}a test.
\end{verbatim}

Additions:

\begin{verbatim}
This {++is ++}a test.
\end{verbatim}

Substitutions:

\begin{verbatim}
This {~~isn't~>is~~} a test.
\end{verbatim}

Highlighting:

\begin{verbatim}
This is a {==test==}.
\end{verbatim}

Comments:

\begin{verbatim}
This is a test{>>What is it a test of?<<}.
\end{verbatim}

\subsection{CriticMarkup Limitations }
\label{criticmarkuplimitations}

If you \texttt{-{}-accept} or \texttt{-{}-reject} CriticMarkup changes, then it should work properly in just about any document.

If you want to try to include your changes as ``notes'' in the final document, then certain situations will lead to results that were probably not what you intended.

\begin{enumerate}
\item CriticMarkup must be contained within a single block (e.g. paragraph, list item, etc.) CM that spans multiple blocks will not be recognized.

\item CriticMarkup that crosses multiple \gls{MMD} spans (e.g. \texttt{\{++** foo\} bar**}) will not properly manage the intended MultiMarkdown markup. This example would not result in bold being applied to \texttt{foo bar}.

\end{enumerate}

\subsection{My philosophy on CriticMarkup}
\label{myphilosophyoncriticmarkup}

I view CriticMarkup as two things (in addition to the actual tools that implement these concepts):

\begin{enumerate}
\item A syntax for documenting editing notes and changes, and for collaborating amongst coauthors.

\item A means to display those notes\slash changes in the HTML output.

\end{enumerate}

I believe that \#1 is a really great idea, and well implemented. \#2 is not so well implemented, largely due to the ``orthogonal'' nature of CriticMarkup and the underlying Markdown syntax.

CM is designed as a separate layer on top of Markdown\slash MultiMarkdown. This means that a Markdown span could, for example, start in the middle of a CriticMarkup structure, but end outside of it. This means that an algorithm to properly convert a CM\slash Markdown document to HTML would be quite complex, with a huge number of edge cases to consider. I've tried a few (fairly creative, in my opinion) approaches, but they didn't work. Perhaps someone else will come up with a better solution, or will be so interested that they put the work in to create the complex algorithm. I have no current plans to do so.

Additionally, there is a philosophical distinction between documenting editing notes, and using those notes to produce a ``finished'' document (e.g. HTML or PDF) that keeps those editing notes intact (e.g. strikethroughs, highlighting, etc.) I believe that CM is incredibly useful for the editing process, but am less convinced for the output process (I know many others disagree with me, and that's ok. And to be clear, I think that what Gabe and Erik have done with CriticMarkup is fantastic!)

There are other CriticMarkup tools besides MultiMarkdown and \href{http://multimarkdown.com/}{MultiMarkdown Composer}\footnote{\href{http://multimarkdown.com/}{http:\slash \slash multimarkdown.com\slash }}, and you are more than welcome to use them.

For now, the \emph{official} MultiMarkdown support for CriticMarkup consists of:

\begin{enumerate}
\item CriticMarkup syntax is ``understood'' by the MultiMarkdown parser, and by MultiMarkdown Composer syntax highlighting.

\item When converting from MultiMarkdown text to an output format, you can ignore CM formatting with \texttt{compatibility} mode (probably not what you want to do), accept all changes, or reject all changes (as above). These are the preferred choices.

\item The secondary choice, is to \emph{attempt} to show the changes in the exported document. Because the syntaxes are orthogonal, this will not always work properly, and will not always give valid output files.

\end{enumerate}

\section{Glossaries }
\label{glossaries}

MultiMarkdown has a feature that allows footnotes to be specified as glossary terms. It doesn't do much for XHTML documents, but the XSLT file that converts the document into LaTeX is designed to convert these special footnotes into glossary entries.

\textbf{NOTE}: The syntax for glossary terms changed in \gls{MMD} v6.

If there are terms in your document you wish to define in a \newglossaryentry{glossary}{name=glossary, description={The glossary collects information about important terms used in your document}}\gls{glossary} at the end of your document, you can define them using the glossary syntax.

Glossary terms can be specified using inline or reference syntax. The inline variant requires that the abbreviation be wrapped in parentheses and immediately follows the \texttt{?}.

\begin{verbatim}
[?(glossary) The glossary collects information about important
terms used in your document] is a glossary term.

[?glossary] is also a glossary term.

[?glossary]: The glossary collects information about important
terms used in your document
\end{verbatim}

Much like abbreviations, there is also a ``shortcut'' method that is similar to the approach used in prior versions of \gls{MMD}. You specify the definition for the glossary term in the usual manner, but \gls{MMD} will automatically identify each instance where the term is used and substitute it automatically. In this case, the term is limited to a more basic character set which includes letters, numbers, periods, and hyphens, but not much else. For more complex glossary terms, you must explicitly mark uses of the term.

\subsection{LaTeX Glossaries }
\label{latexglossaries}

\textbf{Note}: \emph{Getting glossaries to work is a slightly more advanced LaTeX
feature, and might take some trial and error the first few times.}

Unfortunately, it takes an extra step to generate the glossary when creating a
pdf from a latex file:

\begin{enumerate}
\item You need to have the \texttt{basic.gst} file installed, which comes with the
memoir class.

\item You need to run a special makeindex command to generate the \texttt{.glo} file:
\texttt{makeindex -s `kpsewhich basic.gst` -o "filename.gls" "filename.glo"}

\item Then you run the usual pdflatex command again a few times.

\end{enumerate}

Alternatively, you can use the code below to create an engine file for TeXShop (it belongs in \texttt{\ensuremath{\sim}\slash Library\slash TeXShop\slash Engines}). You can name it something like \texttt{MemoirGlossary.engine}. Then, when processing a file that needs a glossary, you typeset your document once with this engine, and then continue to process it normally with the usual LaTeX engine. Your glossary should be compiled appropriately. If you use \href{http://www.uoregon.edu/~koch/texshop/}{TeXShop}\footnote{\href{http://www.uoregon.edu/~koch/texshop/}{http:\slash \slash www.uoregon.edu\slash \ensuremath{\sim}koch\slash texshop\slash }}, this is the way to go.

\begin{verbatim}
#!/bin/	

set path = ($path /usr/local/teTeX/bin/powerpc-apple-darwin-current 
	/usr/local/bin) # This is actually a continuation of the line above

set basefile = `basename "$1" .tex`

makeindex -s `kpsewhich basic.gst` -o "${basefile}.gls" "${basefile}.glo"
\end{verbatim}

\section{Math }
\label{math}

MultiMarkdown 2.0 used \href{http://www1.chapman.edu/~jipsen/mathml/asciimath.html}{ASCIIMathML}\footnote{\href{http://www1.chapman.edu/~jipsen/mathml/asciimath.html}{http:\slash \slash www1.chapman.edu\slash \ensuremath{\sim}jipsen\slash mathml\slash asciimath.html}} to typeset mathematical equations. There
were benefits to using ASCIIMathML, but also some disadvantages.

When rewriting for MultiMarkdown 3.0, there was no straightforward way to
implement ASCIIMathML which lead me to look for alternatives. I settled on
using \href{http://www.mathjax.org/}{MathJax}\footnote{\href{http://www.mathjax.org/}{http:\slash \slash www.mathjax.org\slash }}. The advantage here is that the same syntax is supported by
MathJax in browsers, and in native LaTeX syntax when creating LaTeX documents.

To enable MathJax support in web pages, you have to include a link to an
active MathJax installation --- setting this up is beyond the scope of this
document, but it's not too hard.

Here's an example of the metadata setup, and some math:

\begin{verbatim}
latex input:	mmd-article-header  
Title:			MultiMarkdown Math Example  
latex input:	mmd-article-begin-doc  
latex footer:	mmd-memoir-footer  
HTML header:	<script type="text/javascript" src="https://cdnjs.cloudflare.com/ajax/libs/mathjax/2.7.2/MathJax.js?config=TeX-AMS-MML_HTMLorMML"></script>

		
An example of math within a paragraph --- \\({e}^{i\pi }+1=0\\)
--- easy enough.

And an equation on it's own:

\\[ {x}_{1,2}=\frac{-b\pm \sqrt{{b}^{2}-4ac}}{2a} \\]

That's it.
\end{verbatim}

Here's what it looks like in action (if you're viewing this document in a
supported format):

\begin{quote}
An example of math within a paragraph --- \({e}^{i\pi }+1=0\)
--- easy enough.

And an equation on it's own:

\[ {x}_{1,2}=\frac{-b\pm \sqrt{{b}^{2}-4ac}}{2a} \]

That's it.
\end{quote}

In addition to the \texttt{\textbackslash{}\textbackslash{}[ \textbackslash{}\textbackslash{}]} and \texttt{\textbackslash{}\textbackslash{}( \textbackslash{}\textbackslash{})} syntax, you can use LaTeX-style ``dollar sign'' delimiters:

\begin{verbatim}
An example of math within a paragraph --- ${e}^{i\pi }+1=0$
--- easy enough.

And an equation on it's own:

$${x}_{1,2}=\frac{-b\pm \sqrt{{b}^{2}-4ac}}{2a}$$

That's it.
\end{verbatim}

In order to be correctly parsed as math, there \emph{must} not be any space between the \texttt{\$} and the actual math on the inside of the delimiter, and there \emph{must} be space on the outside. ASCII punctuation can also serve as ``space'' outside of the math.

\subsection{Superscripts and Subscripts }
\label{superscriptsandsubscripts}

You can easily include superscripts and subscripts in MultiMarkdown as well:

\begin{verbatim}
This apartment has an area of 100m^2
One must consider the value of x~z
\end{verbatim}

becomes

\begin{quote}
This apartment has an area of 100m\textsuperscript{2}\\
One must consider the value of x\textsubscript{z}
\end{quote}

The subscript must not contain any whitespace or punctuation.

More complicated exponents and subscripts can be delimited like this:

\begin{verbatim}
y^(a+b)^
x~y,z~
\end{verbatim}

\begin{quote}
y\textsuperscript{(a+b)}\\
x\textsubscript{y,z}
\end{quote}

\section{Metadata }
\label{metadata}

It is possible to include special metadata at the top of a MultiMarkdown
document, such as title, author, etc. This information can then be used to
control how MultiMarkdown processes the document, or can be used in certain
output formats in special ways. For example:

\begin{verbatim}
{{examples/metadata.text}}
\end{verbatim}

The syntax for including metadata is simple.

\begin{itemize}
\item The metadata must begin at the very top of the document - no blank lines can precede it. There can optionally be a \texttt{-{}-{}-} on the line before and after the metadata. The line after the metadata can also be \texttt{...}. This is to provide better compatibility with \href{http://www.yaml.org/}{YAML}\footnote{\href{http://www.yaml.org/}{http:\slash \slash www.yaml.org\slash }}, though MultiMarkdown doesn't support all YAML metadata.

\item Metadata consists of two parts - the \texttt{key} and the \texttt{value}

\item The metadata key must begin at the beginning of the line. It must start with an ASCII letter or a number, then the following characters can consist of ASCII letters, numbers, spaces, hyphens, or underscore characters.

\item The end of the metadata key is specified with a colon (`:')

\item After the colon comes the metadata value, which can consist of pretty much any characters (including new lines). To keep multiline metadata values from being confused with additional metadata, I recommend indenting each new line of metadata. If your metadata value includes a colon, it \emph{must} be indented to keep it from being treated as a new key-value pair.

\item While not required, I recommend using two spaces at the end of each line of metadata. This will improve the appearance of the metadata section if your document is processed by Markdown instead of MultiMarkdown.

\item Metadata keys are case insensitive and stripped of all spaces during processing. This means that \texttt{Base Header Level}, \texttt{base headerlevel}, and \texttt{baseheaderlevel} are all the same.

\item Metadata is processed as plain text, so it should \emph{not} include MultiMarkdown markup.

\item After the metadata is finished, a blank line triggers the beginning of the rest of the document.

\end{itemize}

\section{Metadata ``Variables'' }
\label{metadatavariables}

You can substitute the \texttt{value} for a metadata \texttt{key} in the body of a document using the following format, where \texttt{foo} and \texttt{bar} are the \texttt{key}s of the desired metadata.

\begin{verbatim}
{{examples/metadata-variable.text}}
\end{verbatim}

\section{``Standard'' Metadata keys }
\label{standardmetadatakeys}

There are a few metadata keys that are standardized in MultiMarkdown. You can
use any other keys that you desire, but you have to make use of them yourself.

My goal is to keep the list of ``standard'' metadata keys as short as possible.

\subsection{Author }
\label{author}

This value represents the author of the document and is used in LaTeX, ODF, and RTF
documents to generate the title information.

\subsection{Affiliation }
\label{affiliation}

This is used to enter further information about the author --- a link to a
website, the name of an employer, academic affiliation, etc.

\subsection{Base Header Level }
\label{baseheaderlevel}

This is used to change the top level of organization of the document. For example:

\begin{verbatim}
Base Header Level: 2

# Introduction #
\end{verbatim}

Normally, the Introduction would be output as \texttt{<h1>} in HTML, or \texttt{\textbackslash{}part\{\}} in LaTeX. If you're writing a shorter document, you may wish for the largest division in the document to be \texttt{<h2>} or \texttt{\textbackslash{}chapter\{\}}. The \texttt{Base Header Level} metadata tells MultiMarkdown to change the largest division level to the specified value.

This can also be useful when using transclusion to combine multiple documents.

\texttt{Base Header Level} does not trigger a complete document.

Additionally, there are ``flavors'' of this metadata key for various output formats so that you can specify a different header level for different output formats --- e.g. \texttt{LaTeX Header Level}, \texttt{HTML Header Level}, and \texttt{ODF Header Level}.

If you are doing something interesting with {[File Transclusion]}, you can also use a negative number here. Since metadata is not used when a file is ``transcluded'', this allows you to use a different level of headings when a file is processed on its own.

\subsection{Biblio Style }
\label{bibliostyle}

This metadata specifies the name of the BibTeX style to be used, if you are
not using natbib.

\subsection{BibTeX }
\label{bibtex}

This metadata specifies the name of the BibTeX file used to store citation
information. Do not include the trailing `.bib'.

\subsection{Copyright }
\label{copyright}

This can be used to provide a copyright string.

\subsection{CSS }
\label{css}

This metadata specifies a URL to be used as a CSS file for the produced
document. Obviously, this is only useful when outputting to HTML.

\subsection{Date }
\label{date}

Specify a date to be associated with the document.

\subsection{HTML Header }
\label{htmlheader}

You can include raw HTML information to be included in the \texttt{<head>} section of the document. MultiMarkdown doesn't perform any validation or processing of this data --- it just copies it as is.

As an example, this can be useful to link your document to a working MathJax
installation (not provided by me):

\begin{verbatim}
HTML header:  <script type="text/javascript"
	src="http://example.net/mathjax/MathJax.js">
	</script>
\end{verbatim}

\subsection{HTML Footer }
\label{htmlfooter}

Raw HTML can be included here, and will be appended at the very end of the document, after footnotes, etc. Useful for linking to scripts that must be included after footnotes.

\subsection{Language }
\label{language}

The \texttt{language} metadata key specified the content language for a document using the standardized two letter code (e.g. \texttt{en} for English). Where possible, this will also set the \texttt{quotes language} metadata key to the appropriate value.

\subsection{LaTeX Author }
\label{latexauthor}

Since MultiMarkdown syntax is not processed inside of metadata, you can use the \texttt{latex author} metadata to override the regular author metadata when exporting to LaTeX.

This metadata \emph{must} come after the regular \texttt{author} metadata if it is also being used.

\subsection{LaTeX Begin }
\label{latexbegin}

This is the name of a LaTeX file to be included (via \texttt{\textbackslash{}input\{foo\}}) when exporting to LaTeX. This file will be included after the metadata, and before the body of the document. This is usually where the \texttt{\textbackslash{}begin\{document\}} command occurs, hence the name.

\subsection{LaTeX Config }
\label{latexconfig}

This is a shortcut key when exporting to LaTeX that automatically populates the \texttt{latex leader}, \texttt{latex begin}, and \texttt{latex footer} metadata based on a standardized naming convention.

\texttt{latex config: article} would be the same as the following setup:

\begin{verbatim}
latex leader:	mmd6-article-leader
latex begin:	mmd6-article-begin
latex footer:	mmd6-article-footer
\end{verbatim}

The standard LaTeX support files have been updated to support this naming configuration:

\href{https://github.com/fletcher/MultiMarkdown-6/tree/master/texmf/tex/latex/mmd6}{https:\slash \slash github.com\slash fletcher\slash MultiMarkdown-6\slash tree\slash master\slash texmf\slash tex\slash latex\slash mmd6}

\subsection{LaTeX Footer }
\label{latexfooter}

The name of a file to be included at the end of a LaTeX document.

\subsection{LaTeX Header }
\label{latexheader}

Raw LaTeX source to be added to the metadata section of the document. \textbf{Note}: This is distinct from the \texttt{latex leader}, \texttt{latex begin}, and \texttt{latex footer} metadata which can only contain a filename.

\subsection{LaTeX Leader }
\label{latexleader}

The name of a file to be included at the very beginning of a LaTeX document, before the metadata.

\subsection{LaTeX Mode }
\label{latexmode}

When outputting a document to LaTeX, there are two special options that change
the output slightly --- \texttt{memoir} and \texttt{beamer}. These options are designed to
be compatible with the LaTeX classes of the same names.

\subsection{LaTeX Title }
\label{latextitle}

Since MultiMarkdown syntax is not processed inside of metadata, you can use the \texttt{latex title} metadata to override the regular title metadata when exporting to LaTeX.

This metadata \emph{must} come after the regular \texttt{title} metadata if it is also being used.

\subsection{MMD Header }
\label{mmdheader}

\texttt{MMD Header} provides text that will be inserted before the main body of text, prior to parsing the document. If you want to include an external file, use the transclusion syntax (\texttt{\{\{foo.txt\}\}}).

\subsection{MMD Footer }
\label{mmdfooter}

The \texttt{MMD Footer} metadata is like \texttt{MMD Header}, but it appends text at the end of the document, prior to parsing. Use transclusion if you want to reference an external file.

This is useful for keeping a list of references, abbreviations, footnotes, links, etc. all in a single file that can be reused across multiple documents. If you're building a big document out of smaller documents, this allows you to use one list in all files, without multiple copies being inserted in the master file.

\subsection{ODF Header }
\label{odfheader}

You can include raw XML to be included in the header of a file output in
OpenDocument format. It's up to you to properly format your XML and get it
working --- MultiMarkdown just copies it verbatim to the output.

\subsection{Quotes Language }
\label{quoteslanguage}

This is used to specify which style of ``smart'' quotes to use in the output document. The available options are:

\begin{itemize}
\item dutch (or \texttt{nl})

\item english

\item french (\texttt{fr})

\item german (\texttt{de})

\item germanguillemets

\item spanish (\texttt{es})

\item swedish (\texttt{sv})

\end{itemize}

The default is \texttt{english} if not specified. This affects HTML output. To
change the language of a document in LaTeX is up to the individual.

\texttt{Quotes Language} does not trigger a complete document.

\subsection{Title }
\label{title}

Self-explanatory.

\subsection{Transclude Base }
\label{transcludebase}

When using the {[File Transclusion]} feature to ``link'' to other documents inside a MultiMarkdown document, this metadata specifies a folder that contains the files being linked to. If omitted, the default is the folder containing the file in question. This can be a relative path or a complete path.

This metadata can be particularly useful when using MultiMarkdown to parse a text string that does not exist as a file on the computer, and therefore does not have a parent folder (when using \texttt{stdin} or another application that offers MultiMarkdown support). In this case, the path must be a complete path.

\section{Raw Source }
\label{rawsource}

In older versions of MultiMarkdown you could use an HTML comment to pass raw LaTeX or other content to the final document. This worked reasonably well, but was limited and didn't work well when exporting to multiple formats. It was time for something new.

\gls{MMD} v6 offers a new feature to handle this. Code spans and code blocks can be flagged as representing raw source:

\begin{verbatim}
foo `*bar*`{=html}

```{=latex}
*foo*
```
\end{verbatim}

The contents of the span\slash block will be passed through unchanged.

You can specify which output format is compatible with the specified source:

\begin{itemize}
\item \texttt{html}

\item \texttt{odt}

\item \texttt{epub}

\item \texttt{latex}

\item \texttt{*} -- wildcard matches any output format

\end{itemize}

\section{Smart Typography }
\label{smarttypography}

MultiMarkdown incorporates John Gruber's \href{http://daringfireball.net/projects/smartypants/}{SmartyPants}\footnote{\href{http://daringfireball.net/projects/smartypants/}{http:\slash \slash daringfireball.net\slash projects\slash smartypants\slash }} tool in addition to the core Markdown functionality. This program converts ``plain'' punctuation into ``smarter'' typographic punctuation.

Just like the original, MultiMarkdown converts:

\begin{itemize}
\item Straight quotes (\texttt{"} and \texttt{'}) into ``curly'' quotes

\item Backticks-style quotes (\texttt{``this''}) into ``curly'' quotes

\item Dashes (\texttt{-{}-} and \texttt{-{}-{}-}) into en- and em- dashes

\item Three dots (\texttt{...}) become an ellipsis

\end{itemize}

MultiMarkdown also includes support for quotes styles other than English (the default). Use the \texttt{quotes language} metadata to choose:

\begin{itemize}
\item dutch (\texttt{nl})

\item german(\texttt{de})

\item germanguillemets

\item french(\texttt{fr})

\item spanish(\texttt{en})

\item swedish(\texttt{sv})

\end{itemize}

This feature is enabled by default, but is disabled in \texttt{compatibility} mode, since it is not part of the original Markdown. You can also use the \texttt{-{}-nosmart} command line option to disable this feature.

\section{Table of Contents }
\label{tableofcontents}

As of version 4.7, MultiMarkdown supports the use of \texttt{\{\{TOC\}\}} to insert a Table of Contents in the document. This is automatically generated from the headers included in the document.

When possible, MultiMarkdown uses the ``native'' TOC for a given output format. For example, \texttt{\textbackslash{}tableofcontents} when exporting to LaTeX.

\input{mmd6-tufte-handout-footer}
\end{document}
